\documentclass[a4paper]{article}
\usepackage[utf8]{inputenc}
\usepackage[T2A]{fontenc}
\usepackage[russian]{babel}
\usepackage{graphicx}
\begin{document}

\begin{center}
    Федеральное государственное автономное образовательное учреждение высшего образования\\
	«Национальный исследовательский университет ИТМО»
\end{center}
\vspace{1cm}


\begin{center}
    \large \textbf{Отчет}\\
    \textbf{по лабораторной работе №2}\\
    \large \textbf{«Синтез помехоустойчивого кода»}\\
     по дисциплине «Информатика»\\
	\vspace{1cm}
    Вариант №70\\
\end{center}

\vspace{10cm}
\begin{flushright}
  Выполнила: Богданова Мария Михайловна, группа P3118\\
  Преподаватель: Рыбаков Степан Дмитриевич\\
\end{flushright}

\vspace{1cm}
\begin{center}
    г. Санкт-Петербург\\
    2022г.
\end{center}
\newpage
\tableofcontents
\newpage
\section{Схема декодирования классического кода Хэмминга (7;4)}
\includegraphics[width=15cm]{lab_1/сюданекопать.jpg}
\newpage
\section{1 задача}
1. (52)  1011011 \\
   Биты четности:\\
   1.(1 + 1 + 0 + 1) \% 2 = 1\\
   2.(0 + 1 + 1 + 1) \% 2 = 1\\
   4.(1 + 0 + 1 + 1) \% 2 = 1\\
   7 бит передан с ошибкой \\
   1010 - исходное переданное сообщение. \\ 
\\
Ход решения:
\\
\includegraphics[width=15cm]{lab_1/1решение(52).png}
\\
2. (89) 0101110 \\
   Биты четности: \\
   1.(0 + 0 + 1 + 0) \% 2 = 1\\
   2.(1 + 0 + 1 + 0) \% 2 = 0\\
   4.(1 + 1 + 1 + 0) \% 2 = 1\\
   5 бит передан с ошибкой \\
   0010 - исходное переданное сообщение. \\
\\
Ход решения:
\\
\includegraphics[width=15cm]{lab_1/2решение(89).png}
3. (14) 1111000\\
   Биты четности: \\
   1.(1 + 1 + 0 + 0) \% 2 = 0\\
   2.(1 + 1 + 0 + 0) \% 2 = 0\\
   4.(1 + 0 + 0 + 0) \% 2 = 1\\
   4 бит передан с ошибкой \\
  1000 - исходное переданное сообщение. \\
\\
Ход решения:
\\
\includegraphics[width=15cm]{lab_1/3решение(14).png}
\\
4. (11) 1011000\\
   Биты четности: \\
   1.(1 + 1 + 0 + 0) \% 2 = 0\\
   2.(0 + 1 + 0 + 0) \% 2 = 1\\
   4.(1 + 0 + 0 + 0) \% 2 = 1\\
   6 бит передан с ошибкой. \\
   1010 - исходное переданное сообщение. \\
\\
Ход решения:
\\
\includegraphics[width=15cm]{lab_1/4решение(11).png}
\\
\newpage
\section{Схема декодирования классического кода Хэмминга (15;11)}
\includegraphics[width=15cm]{lab_1/сюдакопать.jpg}
\newpage
\section{2 задача}
(20) 0 1 1 0 0 0 1 0 1 0 0 0 0 0 1 \\
   1. (0 + 1 + 0 + 1 + 1 + 0 + 0 + 1) \% 2 = 0\\
   2. (1 + 1 + 0 + 1 + 0 + 0 + 0 + 1) \% 2 = 0\\
   4. (0 + 0 + 0 + 1 + 0 + 0 + 0 + 1) \% 2 = 0 \\
   8. (0 + 1 + 0 + 0 + 0 + 0 + 0 + 1) \% 2 = 0\\ 
   10011000001 - исходное переданное сообщение.
\\
\\
Ход решения:
\\
\includegraphics[width=15cm]{lab_1/5решение(20)норм.png}
\section{3 задача}
$2^m$ - m - 1 = 744
\\
m =9,52... (ближе к 10)
\\
m = 10 - кол-во проверочных бит
\\
$2^{10}$ - 1 = 1023 - общее кол-во бит ($2^m$-1)
\\
$2^{10}$ - 10 - 1= 1013 - кол-во информационных битов ($2^m$ - m - 1)
\\
коэфф. избыточность - 1013/1023 = 0,99022483
\section{4 доп. задача}
\includegraphics[width=15cm]{lab_1/lastphoto.jpg}
\newpage
\section{Вывод}
В ходе выполнения данной лабораторной работы я узнала про помехоустойчивые коды, научилась декодировать классический код Хэмминга, а также вычислять минимальное число проверочных разрядов и коэффициент избыточности.
\newpage
\section{Список литературы}
1.\textbf{"Помехоустойчивое кодирование. Классификация помехоустойчивых кодов". - Studfile} [Электронный ресурс] - Режим доступа: https://studfile.net/ (Дата обращения: 10.10.22)\\
2.\textbf{"Помехоустойчивое кодирование. Часть 1: код Хэмминга". - Habr} [Электронный ресурс] - Режим доступа: https://habr.com/ru (Дата обращения: 10.10.22)\\
3.\textbf{"Избыточное кодирование, код Хэмминга". - Вики-конспекты ИТМО} [Электронный ресурс] - Режим доступа: https://neerc.ifmo.ru/wiki/ (Дата обращения: 10.10.22)\\
\end{document}
